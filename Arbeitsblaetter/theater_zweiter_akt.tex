\documentclass[11pt, a4paper]{scrartcl}
\usepackage[
	typ=ab,
	fach=Informatik,
]{schule}
\usepackage{mdframed}

\title{Theateraufführung zweiter Akt}

\begin{document}
\section*{Zweiter Akt}

Nach dem ersten Akt geht es für Martin und die Schauspieler weiter im zweiten Akt. Bei diesem benötigt Martin aber viel mehr Rückmeldung von den Schauspielern, damit auch das Licht passend gesteuert wird. Dieses lässt sich Martins Arbeitsbeschreibung entnehmen:

\begin{mdframed}
    Zum Beginn des zweiten Akt ist alles dunkel, nur der RGB-Scheinwerfer ist mit 50\% rot eingeschaltet um die Bühne in eine Morgenstimmung zu versetzen. Der nächste Wechsel des Lichts darf erst dann erfolgen, wenn ein passender Darsteller in Position ist. Da Martin dieses nicht genau einsehen kann, ist dort ein Schalter angebracht, der vom Schauspieler gedrückt wird. Solange der Schalter nicht gedrückt ist, muss Martin warten und kann erst dann den grünen Scheinwerfer einschalten. Diesen kann er dann nach 20 Sekunden wieder ausschalten.

    Anschließend muss Martin für 34 Sekunden die Bühne mit dem RGB-Scheinwerfer in einem Rosa-Farbton (100/0/50) erhellen. Danach schaltet er das komplette Licht aus um für den Abschluss bereit zu sein.

    Dazu müssen mehrere Schauspieler gleichzeitig auf die Bühne kommen, während der gelbe Scheinwerfer aufleuchtet. Damit Martin auch diesen Zeitpunkt genau erfassen kann, weil er wieder nicht alle Schauspieler sehen kann, müssen hier gleichzeitig an drei Stellen jeweils ein Knopf gedrückt werden. Solange dieses nicht der Fall ist muss Martin warten. Diese Schlusseinstellung dauert 50 Sekunden, die mit dem Einschalten der Tageslichtscheinwerfer für den Applaus abgelöst werden. Dieses dauert noch einmal 55 Sekunden.

    Damit ist die Theateraufführung beendet, aber nichts geht über Marketing. Um herauszufinden, wie gut das Stück angekommen ist, hat Martin vier Drehknöpfe unter den Zuschauern verteilt. Damit stellen diese ein, wie gut ihnen das Stück gefallen hat. Den Wert lässt Martin auslesen und stellt den Gesamtwert in Prozent in einem großen Display dar.
\end{mdframed}

\begin{aufgabe}
    Führen Sie das bekannte Modellierungsschema von \textsc{Abbott} für diese neue Situation durch. Schon modellierte Objekte müssen nicht erneut modelliert werden.
\end{aufgabe}

\begin{aufgabe}
    Geben Sie die Besonderheiten im Ablauf gegenüber dem ersten Teil an. Besprechen Sie mit einem Partner die Möglichkeiten für eine Umsetzung.
\end{aufgabe}

\end{document}