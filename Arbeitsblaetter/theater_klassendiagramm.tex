\documentclass[11pt,a4paper,parskip=half]{scrartcl}
\usepackage[
    typ=ib,
    fach=Informatik,
]{schule}

\renewcommand{\ttdefault}{pcr}

\lstset{%
    breaklines=true, % Zeilenumbrüche
    language=Java, % Sprache
    basicstyle=\ttfamily\small,
    morekeywords={def},
    tabsize=2,
    numbers=left, numberstyle=\footnotesize,  numbersep=5pt,
}

\title{Theateraufführung -- Klassendiagramm}

\begin{document}
\section*{Klassendiagramm für die Theateraufführung}

Dieses ist eine mögliche Lösung für das Klassendiagramm der Theateraufführung:

\begin{figure}[ht]
    \centering
    \scalebox{0.95}{
    \begin{tikzpicture}
        \begin{class}[text width = 6.3cm]{Mischpult}{0,0}
            \operation{Mischpult()}
            \operation{reglerAuslesen(reglerNr: Zahl): Zahl}
        \end{class}

        \begin{class}[text width = 3.5cm]{Regler}{8,2}
            \attribute{wert: Zahl}
            \operation{Regler()}
            \operation{gibWert(): Zahl}
            \operation{aendereWert()}
        \end{class}

        \begin{class}[text width = 6.3cm]{RegelbarerRgbScheinwerfer}{0,-4}
            \operation{RegelbarerRgbScheinwerfer()}
            \operation{einschalten()}
            \operation{ausschalten()}
            \operation{aktuallisieren()}
        \end{class}

        \begin{class}[text width = 3.5cm]{RgbScheinwerfer}{8,-6.5}
        \end{class}

        \draw [umlcd style inherit line,->] (Mischpult.north) |- (Regler.west) node[very near end, above]{regler} node[very near end, below]{3};
        \draw [umlcd style inherit line,->] (RegelbarerRgbScheinwerfer.north) -- (Mischpult.south) node[near end, left]{mischpult} node[near end, right]{1};
        \draw [umlcd style inherit line,->] ([yshift=1cm]RegelbarerRgbScheinwerfer.east) -| (RgbScheinwerfer.north) node[very near end, left]{scheinwerfer} node[very near end, right]{1};


        \begin{class}[text width = 3.5cm]{Scheinwerfer}{0,7.5}
        \end{class}

        \begin{class}[text width = 3.5cm]{Helligkeitssensor}{6,7.5}
        \end{class}

        \begin{class}[text width = 14.5cm]{ErweiterteHintergrundbeleuchtung}{3,5}
            \operation{ErweiterteHintergrundbeleuchtung(pin1: Zahl, pin2: Zahl, pin3: Zahl, sonsorPin: Zahl)}
            \operation{einschalten()}
            \operation{ausschalten()}
        \end{class}

        \draw [umlcd style inherit line,->] ([xshift=-3cm]ErweiterteHintergrundbeleuchtung.north) -- (Scheinwerfer.south) node[near end, left]{scheinwerfer} node[near end, right]{3};
        \draw [umlcd style inherit line,->] ([xshift=3cm]ErweiterteHintergrundbeleuchtung.north) -- (Helligkeitssensor.south) node[near end, left]{sensor} node[near end, right]{1};

    \end{tikzpicture}
    }
\end{figure}


\end{document}