\documentclass[11pt, a4paper]{article}
% ================================
%        Pakete
% ================================

\usepackage[dvips, bottom=2.5cm, top=2.5cm, right=2.5cm, left=3cm]{geometry}
\usepackage[utf8]{inputenc}
\usepackage[ngerman]{babel}
\usepackage[babel,german=quotes]{csquotes}
\usepackage[T1]{fontenc}
\usepackage{enumerate}
\usepackage{setspace}
\usepackage{mdframed}
\setstretch{1.3} 
\setlength\parindent{0pt}
\usepackage{textcomp}
\usepackage{enumerate}
\usepackage{color}
\usepackage{graphicx}
\usepackage{colortbl}
\usepackage{hhline}
\usepackage{tabularx}
\usepackage{booktabs}
\usepackage[font={footnotesize}, labelfont=bf]{caption}
\usepackage[dvipsnames]{xcolor}
\usepackage{float}
\usepackage{subfig}
%\usepackage{hhline}
\usepackage{listings}

\renewcommand{\lstlistingname}{Quelltext} 

\definecolor{deepblue}{rgb}{0,0,0.5}
\definecolor{deepred}{rgb}{0.6,0,0}
\definecolor{deepgreen}{rgb}{0,0.5,0}



% ================================
%        Quelltexte JAVA
% ================================

% Java style for highlighting
\newcommand\javastyle{\lstset{
language=Java,
basicstyle=\ttfamily\scriptsize\mdseries,
otherkeywords={self},          
keywordstyle=\ttfamily\scriptsize\mdseries\color{deepblue},
emph={MyClass,__init__},          
emphstyle=\ttfamily\scriptsize\mdseries\color{deepred},   
stringstyle=\ttfamily\scriptsize\mdseries\color{deepgreen},
commentstyle=\ttfamily\scriptsize\mdseries\color{orange},
frame=,                         
showstringspaces=false,
showtabs=true, 
tabsize=3, 
tab=,
showstringspaces=false,
numbers=left,
extendedchars=true,
breaklines=true,
numberstyle=\tiny,
numbersep=9pt,
stepnumber=1,
captionpos=b,
backgroundcolor=\color[gray]{0.95},
}}

\newcommand\javastyleoneline{\lstset{
language=Java,
basicstyle=\ttfamily\scriptsize\mdseries,
otherkeywords={self},          
keywordstyle=\ttfamily\scriptsize\mdseries\color{deepblue},
emph={MyClass,__init__},          
emphstyle=\ttfamily\scriptsize\mdseries\color{deepred},   
stringstyle=\ttfamily\scriptsize\mdseries\color{deepgreen},
commentstyle=\ttfamily\scriptsize\mdseries\color{orange},
frame=,                         
showstringspaces=false,
showtabs=true, 
tabsize=3, 
tab=,
showstringspaces=false,
numbers=left,
extendedchars=true,
breaklines=true,
numberstyle=\tiny,
numbersep=9pt,
stepnumber=1,
captionpos=b,
backgroundcolor=\color[gray]{0.95},
belowcaptionskip=0em,
belowskip=0em,
}}

% Java for external files
\newcommand\javaexternal[2][]{{
\javastyle
\lstinputlisting[#1]{#2}}}

% Java for inline
\newcommand\javainline[1]{{\javastyle\lstinline!#1!}}



% ================================
%        Quelltexte Bash
% ================================

% Bash style for highlighting (ken Abstand unter der eingefügten Bash-Zeile)
\newcommand\bashstyleoneline{\lstset{
language=BASH,
basicstyle=\ttfamily\scriptsize\mdseries,
otherkeywords={self,bash,git},          
keywordstyle=\ttfamily\scriptsize\mdseries\color{deepblue},
emph={MyClass,__init__},          
emphstyle=\ttfamily\scriptsize\mdseries\color{deepred},   
stringstyle=\ttfamily\scriptsize\mdseries\color{deepgreen},
commentstyle=\ttfamily\scriptsize\mdseries\color{orange},
frame=,                         
showstringspaces=false,
showtabs=true, 
tabsize=3, 
tab=,
showstringspaces=false,
numbers=left,
extendedchars=true,
breaklines=true,
numberstyle=\tiny,
numbersep=9pt,
stepnumber=1,
captionpos=b,
backgroundcolor=\color[gray]{0.95},
belowcaptionskip=0em,
belowskip=0em,
}}

% normaler Bash Style
\newcommand\bashstyle{\lstset{
language=BASH,
basicstyle=\ttfamily\scriptsize\mdseries,
otherkeywords={self,bash,git},          
keywordstyle=\ttfamily\scriptsize\mdseries\color{deepblue},
emph={MyClass,__init__},          
emphstyle=\ttfamily\scriptsize\mdseries\color{deepred},   
stringstyle=\ttfamily\scriptsize\mdseries\color{deepgreen},
commentstyle=\ttfamily\scriptsize\mdseries\color{orange},
frame=,                         
showstringspaces=false,
showtabs=true, 
tabsize=3, 
tab=,
showstringspaces=false,
numbers=left,
extendedchars=true,
breaklines=true,
numberstyle=\tiny,
numbersep=9pt,
stepnumber=1,
captionpos=b,
backgroundcolor=\color[gray]{0.95},
}}

% Bash Style in schwarz
\newcommand\bashstyleblack{\lstset{
language=BASH,
basicstyle=\ttfamily\scriptsize\mdseries,
otherkeywords={self,bash,git},          
keywordstyle=\ttfamily\scriptsize\mdseries\color{black},
emph={MyClass,__init__},          
emphstyle=\ttfamily\scriptsize\mdseries\color{black},   
stringstyle=\ttfamily\scriptsize\mdseries\color{black},
commentstyle=\ttfamily\scriptsize\mdseries\color{black},
frame=,                         
showstringspaces=false,
showtabs=true, 
tabsize=3, 
tab=,
showstringspaces=false,
numbers=left,
extendedchars=true,
breaklines=true,
numberstyle=\tiny,
numbersep=9pt,
stepnumber=1,
captionpos=b,
backgroundcolor=\color[gray]{0.95}
}}

% BASH for external files
\newcommand\bashexternal[2][]{{
\bashstyle
\lstinputlisting[#1]{#2}}}

% BASH for inline
\newcommand\bashinline[1]{{\bashstyle\lstinline!#1!}}


\begin{document}
\pagestyle{empty}
\textbf{{\Large Die Lichtanlage für eine Theateraufführung -- Teil 2}}\\

Hier siehst du, wie Martin weiterarbeitet:

\begin{mdframed}
Dann arbeitet Martin mit den Reglern, die an einen AD-Wandler angeschlossen sind. Er möchte, um flexibler auf die kommenden Szenen reagieren zu können, gerne das Licht mit dem mit den Reglern einstellen. Dazu stehen ihm drei Regler bereit, um Werte für die drei Farbkanäle des RGB-Scheinwerfers einzustellen.\\

Zunächst dreht er alle drei Regler voll auf. Damit ist das Licht für die aktuelle Szene eingestellt. Dann wartet Martin, verändert aber währddessen die Stellung der Regler, indem er den grünen auf ca. 80\% und den roten auf ca 50\% dreht. Dann stellt er die Farbe ein und schaltet den grünen Scheinwerfer zur Verstärkung an. Nach 5 Sekunden schaltet er zur Verstärkung des roten Lichts auch den roten Scheinwerfer ein. Nach weiteren fünf Sekunden (in der Zwischenzeit hat Martin die Stellung des grünen und blauen Reglers wieder ein bisschen verändert) stellt er wieder die neue Farbe des RGB-Scheinwerfers ein.\\

Nach 10 Sekunden schaltet Martin alle Lichter aus und möchte mit dem Stroboskop arbeiten. Er startet das endlose Blinken des weißen Scheinwerfers und stellt zunächst eine Blinkfrequenz von 180 Millisekunden ein. Nach 5 Sekunden ändert er die Frequenz auf 140 Millisekunden, nach weiteren 5 Sekunden auf 90 Millisekunden und dann auf 35 Millisekunden. Dann wartet Martin auf das Signal der Schauspieler, das Stroboskop abzuschalten: Diese haben an der Bühnendekoration einen Taster angebracht, der gedrückt werden muss, um das Signal zum Abschalten zu geben. Martin fragt also beim Taster nach und schaltet abhängig von der Antwort das Stroboskop ab oder lässt es noch ein bisschen länger laufen, bis er erneut beim Taster  nachfragt.
\end{mdframed}

\vfill
\textbf{Aufgaben}:
\begin{enumerate}
\item Führen Sie das schon bekannte Modellierungsschema von \textsc{Abbott} für diese neue Situation durch. Schon modellierte Objekte brauchen Sie natürlich nicht noch einmal zu modellieren, sondern können auf Ihre Unterlagen zurückgreifen.
\end{enumerate}

\end{document}