\documentclass[11pt, a4paper]{article}
% ================================
%        Pakete
% ================================

\usepackage[dvips, bottom=2.5cm, top=2.5cm, right=2.5cm, left=3cm]{geometry}
\usepackage[utf8]{inputenc}
\usepackage[ngerman]{babel}
\usepackage[babel,german=quotes]{csquotes}
\usepackage[T1]{fontenc}
\usepackage{enumerate}
\usepackage{setspace}
\usepackage{mdframed}
\setstretch{1.3} 
\setlength\parindent{0pt}
\usepackage{textcomp}
\usepackage{enumerate}
\usepackage{color}
\usepackage{graphicx}
\usepackage{colortbl}
\usepackage{hhline}
\usepackage{tabularx}
\usepackage{booktabs}
\usepackage[font={footnotesize}, labelfont=bf]{caption}
\usepackage[dvipsnames]{xcolor}
\usepackage{float}
\usepackage{subfig}
%\usepackage{hhline}
\usepackage{listings}

\renewcommand{\lstlistingname}{Quelltext} 

\definecolor{deepblue}{rgb}{0,0,0.5}
\definecolor{deepred}{rgb}{0.6,0,0}
\definecolor{deepgreen}{rgb}{0,0.5,0}



% ================================
%        Quelltexte JAVA
% ================================

% Java style for highlighting
\newcommand\javastyle{\lstset{
language=Java,
basicstyle=\ttfamily\scriptsize\mdseries,
otherkeywords={self},          
keywordstyle=\ttfamily\scriptsize\mdseries\color{deepblue},
emph={MyClass,__init__},          
emphstyle=\ttfamily\scriptsize\mdseries\color{deepred},   
stringstyle=\ttfamily\scriptsize\mdseries\color{deepgreen},
commentstyle=\ttfamily\scriptsize\mdseries\color{orange},
frame=,                         
showstringspaces=false,
showtabs=true, 
tabsize=3, 
tab=,
showstringspaces=false,
numbers=left,
extendedchars=true,
breaklines=true,
numberstyle=\tiny,
numbersep=9pt,
stepnumber=1,
captionpos=b,
backgroundcolor=\color[gray]{0.95},
}}

\newcommand\javastyleoneline{\lstset{
language=Java,
basicstyle=\ttfamily\scriptsize\mdseries,
otherkeywords={self},          
keywordstyle=\ttfamily\scriptsize\mdseries\color{deepblue},
emph={MyClass,__init__},          
emphstyle=\ttfamily\scriptsize\mdseries\color{deepred},   
stringstyle=\ttfamily\scriptsize\mdseries\color{deepgreen},
commentstyle=\ttfamily\scriptsize\mdseries\color{orange},
frame=,                         
showstringspaces=false,
showtabs=true, 
tabsize=3, 
tab=,
showstringspaces=false,
numbers=left,
extendedchars=true,
breaklines=true,
numberstyle=\tiny,
numbersep=9pt,
stepnumber=1,
captionpos=b,
backgroundcolor=\color[gray]{0.95},
belowcaptionskip=0em,
belowskip=0em,
}}

% Java for external files
\newcommand\javaexternal[2][]{{
\javastyle
\lstinputlisting[#1]{#2}}}

% Java for inline
\newcommand\javainline[1]{{\javastyle\lstinline!#1!}}



% ================================
%        Quelltexte Bash
% ================================

% Bash style for highlighting (ken Abstand unter der eingefügten Bash-Zeile)
\newcommand\bashstyleoneline{\lstset{
language=BASH,
basicstyle=\ttfamily\scriptsize\mdseries,
otherkeywords={self,bash,git},          
keywordstyle=\ttfamily\scriptsize\mdseries\color{deepblue},
emph={MyClass,__init__},          
emphstyle=\ttfamily\scriptsize\mdseries\color{deepred},   
stringstyle=\ttfamily\scriptsize\mdseries\color{deepgreen},
commentstyle=\ttfamily\scriptsize\mdseries\color{orange},
frame=,                         
showstringspaces=false,
showtabs=true, 
tabsize=3, 
tab=,
showstringspaces=false,
numbers=left,
extendedchars=true,
breaklines=true,
numberstyle=\tiny,
numbersep=9pt,
stepnumber=1,
captionpos=b,
backgroundcolor=\color[gray]{0.95},
belowcaptionskip=0em,
belowskip=0em,
}}

% normaler Bash Style
\newcommand\bashstyle{\lstset{
language=BASH,
basicstyle=\ttfamily\scriptsize\mdseries,
otherkeywords={self,bash,git},          
keywordstyle=\ttfamily\scriptsize\mdseries\color{deepblue},
emph={MyClass,__init__},          
emphstyle=\ttfamily\scriptsize\mdseries\color{deepred},   
stringstyle=\ttfamily\scriptsize\mdseries\color{deepgreen},
commentstyle=\ttfamily\scriptsize\mdseries\color{orange},
frame=,                         
showstringspaces=false,
showtabs=true, 
tabsize=3, 
tab=,
showstringspaces=false,
numbers=left,
extendedchars=true,
breaklines=true,
numberstyle=\tiny,
numbersep=9pt,
stepnumber=1,
captionpos=b,
backgroundcolor=\color[gray]{0.95},
}}

% Bash Style in schwarz
\newcommand\bashstyleblack{\lstset{
language=BASH,
basicstyle=\ttfamily\scriptsize\mdseries,
otherkeywords={self,bash,git},          
keywordstyle=\ttfamily\scriptsize\mdseries\color{black},
emph={MyClass,__init__},          
emphstyle=\ttfamily\scriptsize\mdseries\color{black},   
stringstyle=\ttfamily\scriptsize\mdseries\color{black},
commentstyle=\ttfamily\scriptsize\mdseries\color{black},
frame=,                         
showstringspaces=false,
showtabs=true, 
tabsize=3, 
tab=,
showstringspaces=false,
numbers=left,
extendedchars=true,
breaklines=true,
numberstyle=\tiny,
numbersep=9pt,
stepnumber=1,
captionpos=b,
backgroundcolor=\color[gray]{0.95}
}}

% BASH for external files
\newcommand\bashexternal[2][]{{
\bashstyle
\lstinputlisting[#1]{#2}}}

% BASH for inline
\newcommand\bashinline[1]{{\bashstyle\lstinline!#1!}}


\begin{document}
\textbf{{\Large Die Lichtanlage für eine Theateraufführung}}\\

Für das Schultheater-Aufführung des Stückes \enquote{Mac Bath} von Shakebeer soll mit Hilfe des Raspberry Pis die Lichtanlage gesteuert werden. Alle Scheinwerfer sind über der Bühne an einem Gerüst angebracht und können ihre Position nicht verändern. Martin ist für das Licht verantwortlich. \\

Hier ist ein Ausschnitt seiner Arbeit:
\begin{mdframed}
Martin schaltet zunächst den grünen, blauen und roten Scheinwerfer an. Nachdem er etwa 20 Sekunden gewartet hat, schaltet er auch den weißen Scheinwerfer an und schaltet den grünen aus. Nach weiteren  10 Sekunden lässt er die gelbe Lampe kurz blinken und schaltet dann das gesamte Licht aus. Für die nächste Szene wird die spezielle Hintergrundbeleuchtung aktiviert. Dazu wird ein Helligkeitssenor befragt und mit der Antwort die Hintergrundbeleuchtung geschaltet.\\

Dann arbeitet Martin mit dem RGB-Scheinwerfer\footnote{Der RGB-Scheinwerfer kann -- anders als die anderen -- eine beliebige Farbe annehmen. Dazu stellt man den Anteil an Rot, Grün und Blauwert ein}: Er stellt eine Farbe von 50\% für grün, 20\% für rot und 25\% für blau ein. Nach 10 Sekunden stellt er eine neue Farbe ein: 10\% für grün, 70\% für rot und 65\% für blau. Dann wartet er nochmal 10 Sekunden und schaltet das Licht ab.\\

Um einen Farbwechsel zu erzielen, lässt Martin dann die Bühne mehrere Male für jeweils fünf Sekunden in weißes, rotes und blaues Licht hüllen. Die Gesamtdauer des Farbwechsels dauert ca. 45 Sekunden.\\
\end{mdframed}

\newpage
\textbf{Aufgaben}:
\begin{enumerate}
\item Entwerfen Sie zu der gegebenen Problembeschreibung mit Hilfe des Verfahrens von Abbott ein objektorientiertes Modell, indem Sie die relevanten Objekte mit ihren Attributen und Methoden identifizieren. Notieren Sie die Objekte als Objektkarten.

%\item Informieren Sie sich über die Möglichkeiten eines ******* und wofür er für obige Anforderungen im Programm benötigt wird.

\item Informieren Sie sich über die Möglichkeiten eines AD-Wandlers und wofür er für obige Anforderungen im Programm benötigt wird.

\item Erstellen Sie ein Sequenzdiagramm auf Folie, welches die Interaktion zwischen den identifizierten Objekten gemäß der Problembeschreibung beschreibt und bereiten Sie Lösung so vor, dass Sie diese vorstellen können:
\begin{enumerate}
	\item für den Scheinwerfer
	\item für den Helligkeitssenor
	\item für den RGB-Scheinwerfer
	\item für das Stroboskop
\end{enumerate}
\end{enumerate}

\vfill
\emph{Bauteile}:
\begin{itemize}
\item Diode
\item Phototransistor
\item RGB-LED
\item AD-Wandler
\end{itemize}

%\vspace{0.5cm}
%\emph{Erweiterbar um}:
%\begin{itemize}
%\item Taster (Bei Taster-Druck blinken alle Lampen oder alle Lampen gehen aus)
%\item Phototransistor (wenn es zu hell draußen wird (Umfeld Party), soll das Licht im Raum automatisch aus gehen)
%\end{itemize}

\end{document}