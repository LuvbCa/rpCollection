\documentclass[11pt, a4paper]{scrartcl}
\usepackage[
	typ=ab,
	fach=Informatik,
]{schule}
\usepackage{mdframed}

\title{Die Lichtanlage einer Theater-Aufführung -- Teil 2}

\begin{document}
\section*{Die Lichtanlage einer Theater-Aufführung -- Teil 2}

Hier siehst du, wie Martin weiterarbeitet:

\begin{mdframed}
Für die nächsten Szene trägt Martin den RGB-Scheinwerfer auf die Empore und Martin das Licht gerne flexibel regeln. Dazu stehen ihm drei Regler zur Verfügung, um die Farbe zu steuern. Er hat für jeden Farbkanal einen Regler.\\

Zunächst dreht er alle drei Regler voll auf. Dann gibt er den Befehl an den RGB-Scheinwerfer, die eingestellten Werte anzunehmen. Während auf der Bühne das Theaterstück seinen Lauf nimmt, verändert er die Regler. Er verstellt den grünen auf ca. 80\%, den roten auf ca 50\% und dreht den blauen auf 0\%. Nach 10 Sekunden nimmt dann der RGB-Scheinwerfer die eingestellte Farbe an.\\

Dann schleicht sich Martin hinter die Bühne und baut die Hintergrundbeleuchtung ab (er überprüft zuvor noch einmal, ob die Hintergrundbeleuchtung auch wirklich ausgeschaltet ist).

(Objekt auflösen?)\\

Für die kommende Szene arbeitet Martin mit den Schauspielern auf der Bühne zusammen. Beim RGB-Scheinwerfer hat er mittlerweile ein leichtes rosa eingestellt. Die Schauspieler haben eine kleine Taschenlampe in der Hand und wollen den roten Scheinwerfer immer dann anschalten, wenn sie mit der Taschenlampe auf den Helligkeitssensor scheinen.\\

Für das Finale wartet Martin dann auf das Signal der Schauspieler, alle Lampen auf einmal anzuschalten. Dafür haben die Schauspieler an der Bühnendekoration einen geheimen Taster angebracht, der gedrückt werden muss, um das Signal zum Einschalten zu geben. Martin fragt also beim Taster nach und schaltet abhängig von der Antwort dann alle Scheinwerfer. Nach 5 Sekunden schaltet er alle Lichter aus. Dann räumt Martin alle Scheinwerfer wieder in den Medienraum. Das Theaterstück ist zu Ende. 
\end{mdframed}

\vfill

\begin{aufgabe}
Führen Sie das schon bekannte Modellierungsschema von \textsc{Abbott} für diese neue Situation durch. Schon modellierte Objekte brauchen Sie natürlich nicht noch einmal zu modellieren, sondern können auf Ihre Unterlagen zurückgreifen.
\end{aufgabe}

\end{document}